% Nothing in Introduction
\addtocounter{section}{1}

\section{Core and Equilibria}

\begin{defn}
    A coalition $S \subset I$ \emph{improves upon}, or \emph{blocks}, the feasible allocation $x^* = (x^*_1, \dots, x^*_I) \in \mathbb{R}^{LI}_+$ if for every $i \in S$ we can find a consumption $x_i \geq 0$ with the properties:
    \begin{enumerate}
        \item $x_i \succ_i x_i^*$ for every $i \in S$
        \item $\sum_{i \in S} x_i \in Y + \left\{\sum_{i \in S} \omega_i \right\}$.
    \end{enumerate}
\end{defn}

\begin{defn}
    We say that a feasible allocation $x^* = (x^*_1, \dots, x^*_I) \in \mathbb{R}^{LI}_+$ has the \emph{core property} if there is no coalition of consumers $S \subset I$ that can improve upon $x^*$. The \emph{core} is the set of allocations that have the core property.
\end{defn}

\begin{prop}
    Any Walrasian equilibrium allocation has the core property.
\end{prop}

\begin{prop}
    Denoting by $hn$ the $n$th individual of type $h$, suppose that the allocation
    \begin{equation*}
        x^* = (x^*_{11}, \dots, x^*_{1n}, \dots, x^*_{1N}, \dots, x^*_{H1}, \dots, x^*_{Hn}, \dots, x^*_{HN}) \in \mathbb{R}^{LHN}_+
    \end{equation*}
    belongs to the core of the $N$-replica economy. Then $x^*$ has the \emph{equal-treatment property}, that is, all consumers of the same type get the same consumption bundle:
    \begin{equation*}
        x^*_{hm} = x^*_{hn} \quad \text{for all } 1 \leq m, n \leq N \text{ and } 1 \leq h \leq H.
    \end{equation*}
\end{prop}

\begin{prop}
    If the feasible type allocation $x^* = (x^*_1, \dots, x^*_H) \in \mathbb{R}^{LH}_+$ has the core property for all $N = 1, 2, \dots$, that is, $x^* \in C_N$ for all $N$, then $x^*$ is a Walrasian equilibrium allocation.
\end{prop}


\section{Noncooperative Foundations of Walrasian Equilibria}

\begin{defn}
    The profiles of actions $a^* = (a^*_1, \dots, a^*_I) \in A_1 \times \dots \times A_I$ is a \emph{trading equilibrium} if, for every $i$,
    \begin{equation*}
        u_i \left( g\left( a^*_i; p(a^*)\right) + \omega_i \right) \geq u_i \left( g\left( a_i; p(a_i; a^*_{-i})\right) + \omega_i \right) \quad \text{for all } a_i \in A_i.
    \end{equation*}
\end{defn}


\section{The Limits to Redistribution}

\begin{defn}
    The feasible allocation $x^* = (x^*_1, \dots, x^*_I) \in \mathbb{R}^{LI}_+$ is \emph{self-selective} (or \emph{anonymous}, or \emph{envy-free in net trades}) if there is a set of net trades $B \subset \mathbb{R}^{L}$, to be called a \emph{generalised budget set}, or a \emph{tax system}, such that, for every $i$, $z^*_i = x^*_i - \omega_i$ solves the problem
    \begin{align*}
        \max \, &u_i(z_i + \omega_i) \\
        \text{s.t. } &z_i \in B, \\
        &z_i + \omega_i \geq 0.
    \end{align*}
\end{defn}

\begin{prop}
    Suppose we have an exchange economy with a continuum of consumer types. Assume:
    \begin{enumerate}
        \item The preferences of all consumers are representable by differentiable utility functions.
        \item The set of characteristics of consumers present in the economy cannot be split into two disconnected classes. Formally, if $\left(u(\cdot), \omega), (u'(\cdot), \omega' \right)$ are two preferences-endowment pairs present in the economy then there is a continuous function $(u(\cdot; t), \omega(t))$ of $t \in [0, 1]$ such that
        \begin{equation*}
            (u(\cdot; 0), \omega(0)) = (u(\cdot, \omega)), (u(\cdot; 1), \omega(1)) = (u'(\cdot), \omega),
        \end{equation*}
        and $(u(\cdot; t), \omega(t))$ is present in the economy for every $t$.
    \end{enumerate}

    Then any allocation $x^* = \{x^*_i\}_{i \in I}$ that is Pareto optimal, self-selective, and interior (i.e., $x^*_i \gg 0$ for all $i$) must be a Walrasian equilibrium allocation. Here $I$ is an infinite set of names.
\end{prop}


\section{Equilibrium and the Marginal Productivity Principle}

\begin{defn}
    Given a continuum population $\mu = (\mu_1, \dots, \mu_H) \in \mathbb{R}^{H}_+$ a feasible allocation $(x^*_1, \dots, x^*_H)$ is a \emph{marginal product}, or \emph{no-surplus}, \emph{allocation} if
    \begin{equation*}
        u_h (x^*_h) = \frac{\partial v(\mu)}{\partial \mu_h} \quad \text{for all } h.
    \end{equation*}
    In words: at a no-surplus allocation everyone is getting exactly what she contributes on the margin.
\end{defn}

\begin{prop}
    For any \emph{continuum} population $\bar{\mu} = (\bar{\mu}_1, \dots, \bar{\mu}_H) \gg 0$ a feasible allocation $(x^*_1, \dots, x^*_H) \gg 0$ is a marginal product allocation if and only if it is a Walrasian equilibrium allocation.
\end{prop}
